% Beware: Commenting out, changing or removing lines is only allowed where explicitly stated!

% Set document variables
\newcommand{\documentLanguage}{de} % or f.i. en

\newcommand{\projectTitle}{Overkomplex Title}
\newcommand{\projectType}{T2000}
\newcommand{\authorOne}{Max Musterman}
\newcommand{\submissionDate}{8. September 2025}
\newcommand{\projectTimeframe}{3 Monate}
\newcommand{\matriculationNumber}{5321326}
\newcommand{\course}{INF23A}
\newcommand{\company}{TRUMPF SE + Co. KG}
\newcommand{\companyLocation}{Ditzingen}
\newcommand{\tutor}{Betreuer}
\newcommand{\secondTutor}{}
\newcommand{\studyProgram}{Informatik}
\newcommand{\dhbw}{Stuttgart}
\newcommand{\headingTopLeft}{\dhbw}

% Set heading and misc names
\addto\captionsenglish{% Replace "english" with the language you use
  \renewcommand{\contentsname}{Table of contents}%
}
\renewcommand{\listfigurename}{Abbildungsverzeichnis}
\renewcommand{\figurename}{Abbildung}
\renewcommand{\listtablename}{Tabellenverzeichnis}
\renewcommand{\tablename}{Tabelle}
\newcommand{\tableOfAcronymsName}{Abkürzungsverzeichnis}
\newcommand{\tableOfUnitsName}{Einheitenverzeichnis}
\renewcommand{\lstlistlistingname}{Quellcodeverzeichnis}
\renewcommand{\lstlistingname}{Quellcode}
\newcommand{\equationName}{Formel}
\newcommand{\listOfEquationsName}{Formelverzeichnis}
\newcommand{\userFrameName}{Rahmen}
\newcommand{\listOfUserFramesName}{Rahmenverzeichnis}
\definecolor{userFrameColor}{RGB}{218, 219, 243}
\newcommand{\bibliographyName}{Literaturverzeichnis}
\newcommand{\printMediaTitle}{Printquellen}
\newcommand{\webMediaTitle}{Internetquellen}
\newcommand{\BIBand}{und}
\newcommand{\appendixName}{Anhang}
\newcommand{\toDosName}{To-Do's}

\newcommand{\mainLogo}{media/DHBW_Logo.png}
\newcommand{\confidentialText}{Vertraulich}
\newcommand{\projectTimeframeText}{Bearbeitungszeitraum}
\newcommand{\matriculationNumberText}{Matrikelnummer}
\newcommand{\courseText}{Kurs}
\newcommand{\companyText}{Ausbildungsfirma}
\newcommand{\tutorText}{Betreuer}
\newcommand{\secondTutorText}{Zweitbetreuer}
\newcommand{\tableContinuationText}{Fortsetzung auf der nächsten Seite}
\newcommand{\tableContinuationTextPos}{l} % Can be set to l or c or r 
\newcommand{\tableContinuationCaptionText}{(Fortsetzung)}
\newcommand{\watermarkText}{Wasserzeichen}

% Custom bib classification
\defbibfilter{printMediaFilter}{
  type=book or type=article
}
\defbibfilter{webMediaFilter}{
  not type=book and not type=article
}

% Enable/Disable an option by activating/commenting the according line:
% \newcommand{\listofunitsVisible}{}
\newcommand{\listofacronymsVisible}{}
\newcommand{\listoffiguresVisible}{}
% \newcommand{\listoftablesVisible}{}
% \newcommand{\listoflistedEquationsVisible}{}
\newcommand{\listoflistingsVisible}{}
% \newcommand{\listofuserFramesVisible}{}
\newcommand{\secondLogo}{media/TRUMPF_Logo}
% \newcommand{\watermarkActive}{}
% \newcommand{\restrictedActive}{}
% \newcommand{\alternateHeader}{}
\newcommand{\toDosActive}{}